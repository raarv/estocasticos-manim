\documentclass{beamer}
\usepackage[skip=10pt plus1pt, indent=40pt]{parskip}
%Information to be included in the title page:
\title{Black Scholes}
\author{Fernanda, Rafael y Fernanda}
\institute{ITAM}
\date{2022}

\begin{document}

\frame{\titlepage}

\begin{frame}
A continuación explicaremos a mayor detalle cómo funciona Black-Scholes.
Comenzaremos con una acción X, para la cual el valor presente es:
\begin{equation*}
X(0) = x_{0}
\end{equation*}
a tiempo t, su precio es:
\begin{equation*}
X(t)
\end{equation*}
y asumimos que tenemos un factor de descuento $\alpha$, por lo que el valor presente de la acción a tiempo t es
\begin{equation*}
	e^{-\alpha t}X(t)
\end{equation*}
\end{frame}

\begin{frame}
	Estamos interesados en el comportamiento de esta acción en el intervalo de tiempo $[0,T]$\par
	Suponiendo que tenemos opciones compradas a tiempo 0, la opción tiene costo c por acción, lo que nos dará la opción de comprar acciones a tiempo t, por el precio fijo de K por cada una de las acciones.\par
	Lo que buscamos con Black-Scholes es determinar el precio c tal que no se tenga arbitraje
\end{frame}

\begin{frame}
	De esto se sigue que no habrá una estrategia financiera que garantice un retorno positivo si y solo si existe una medida de probabilidad P sobre el conjunto de posibilidades tal que el total de las apuestas tenga esperanza de retorno 0

\begin{equation} \label{eq:1}
\mathbb{E}_{\textbf{P}}[e^{-\alpha t} X(t) | X(u), 0 \leq u \leq s] = e^{-\alpha s} X(s)
\end{equation}
\end{frame}

\begin{frame}
Dado que tenemos un call a tiempo t por un precio K, el valor de la opción al tiempo t está dado por:
\begin{align*}
	X(t) - K, \quad &\text{si } X(t) \ge K\\
        0,  \quad &\text{si } X(t) < K
\end{align*}
Escrito de otra manera, el valor de la opción a tiempo t es:
\begin{equation*}
\left( X(t) - K \right)^{+}
\end{equation*}
Por lo tanto, el valor presente de la opción es:
\begin{equation*}
e^{-\alpha t}\left( X(t) - K \right)^{+}
\end{equation*}
\end{frame}

\begin{frame}
Si c es el costo de la opción a tiempo 0, y queremos tener un retorno de la opción esperado igual a 0, necesitamos que:
\begin{equation} \label{eq:2}
\mathbb{E}_{\textbf{P}}[e^{-\alpha t}\left( X(t) - K \right)^{+}] = c
\end{equation}
\end{frame}

\begin{frame}
	Por el teorema de arbitraje, si podemos encontrar una medida de probabilidad $\textbf{P}$ en el conjunto de resultados que satisfaga la ecuación \ref{eq:1}, y si c está dado como en la ecuación \ref{eq:2}, entonces no es posible que haya arbitraje.
\end{frame}

\begin{frame}
A continuación presentamos una medida de probabilidad $\textbf{P}$ en el resultado $X(t),\quad 0 \leq t \leq T$ que satisface la ecuación \ref{eq:1}\par
Supongamos que
\begin{equation*}
	X(t) = x_{0}e^{Y(t)}
\end{equation*}
Donde $\{Y(t),\ t\geq 0\}$ es un proceso de movimiento Browniano con coeficiente de deriva $\mu$ y varianza $\sigma^2$. Por lo que $\{X(t),\ t\geq 0\}$ es un proceso Browniano geométrico.
\end{frame}

\end{document}
